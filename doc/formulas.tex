\documentclass[letterpaper]{article}
\usepackage{psiform}
\newcommand*\Tdb{\ensuremath{T\sub{db}}}
\newcommand*\Twb{\ensuremath{T\sub{wb}}}
\newcommand*\Tdp{\ensuremath{T\sub{dew}}}
\newcommand*\Pw{\ensuremath{P\sub{w}}}
\newcommand*\Pws{\ensuremath{P\sub{ws}}}
\newcommand*\RH{\ensuremath{w\sub{RH}}}
\newcommand*\Rda{\ensuremath{\rho\sub{da}}}
\title{Análisis de funciones del programa \texttt{psychropy}}
\author{Chlewey}
\date{2015}
\begin{document}
\maketitle
\begin{abstract}
aaa
\end{abstract}
\tableofcontents
\section{Fórmulas de \texttt{psychropy}}

\subsection{\funcname{Part\_press}[P,W]}
\noindent Parámetros:
\vardef{P}presión ambiental [kPa]
\vardef{W}humedad específica [kg/kg dry air]
\\Salida:
\vardef{\Pw}presión parcial del vapor [kPa]
\ashref, page 6.9 equation 38

\begin{equation}
\Pw=\frac{P\cdot W}{0.62198 + W}
\end{equation}


\subsection{\funcname{Sat\_press}[\Tdb]}
\noindent Parámetros:
\vardef{\Tdb}temperatura de bulbo seco [\degC] (válido entre -100\degC\ y 200\degC)
\\Salida:
\vardef{\Pws}presión parcial del vapor [kPa]
\ashref, p 6.2, equation 5 and 6

\begin{equation}
    T = \Tdb + 273.15                     %# Converts from degC to degK
\end{equation}    
Si $\Tdb\le 0$:
\begin{align}
  c_1 &= -5674.5359\\
  c_2 &= 6.3925247\\
  c_3 &= -0.009677843\\
  c_4 &= 0.00000062215701\\
  c_5 &= 2.0747825\times10^{-9}\\
  c_6 &= -9.484024\times10^{-13}\\
  c_7 &= 4.1635019\\
  \Pws &= \frac1{1000}\left(\frac{c_1}{T_K} + c_2 + c_3T_K + c_4T_K^2 + c_5T_K^3 + c_6T_K^4 + c_7\ln{T_K}\right)
\end{align}
Si no:
\begin{align}
  c_8 &= -5800.2206\\
  c_9 &= 1.3914993\\
  c_{10} &= -0.048640239\\
  c_{11} &= 0.000041764768\\
  c_{12} &= -0.000000014452093\\
  c_{13} &= 6.5459673\\
  \Pws &= \frac1{1000}\left(\frac{c_8}{T_K} + c_9 + c_{10}T_K + c_{11}T_K^2 + c_{12}T_K^3 + c_{13}\ln{T_K}\right)
\end{align}

\subsection{\funcname{Hum\_rat}[\Tdb, \Twb, P]}
\noindent Parámetros:
\vardef{\Tdb}temperatura de bulbo seco [\degC]
\vardef{\Twb}temperatura de bulbo humedo [\degC]
\vardef{P}presión ambiental [kPa]
\\Salida:
\vardef{W}humedad específica [kg/kg dry air]
\ashref.

\begin{align}
    \Pws &= \funcname{Sat\_press}[\Twb] \\
    W\sub s &= \frac{0.62198 \cdot \Pws}{P - \Pws}          %# Equation 23, p6.8
\end{align}
Si $\Tdb\ge 0$:
\begin{equation}
W = \frac{(2501 - 2.326\Twb)W\sub s - 1.006(\Tdb - \Twb)}{2501 + 1.86\Tdb - 4.186\Twb}
\end{equation}
Si no:
\begin{equation}
W = \frac{(2830 - 0.24\Twb)W\sub s - 1.006(\Tdb - \Twb)}{2830 + 1.86\Tdb - 2.1\Twb}
\end{equation}

\subsection{\funcname{Hum\_rat2}[\Tdb, \RH, P]}
\noindent Parámetros:
\vardef{\Tdb}temperatura de bulbo seco [\degC]
\vardef{\RH}humedad relativa [fracción o porcentaje]
\vardef{P}presión ambiental [kPa]
\\Salida:
\vardef{W}humedad específica [kg/kg dry air]
\ashref.

\begin{align}
    \Pws &= \funcname{Sat\_press}[\Tdb] \\
    W &= \frac{0.62198\RH\cdot \Pws}{P - \RH\cdot \Pws}
\end{align}

\subsection{\funcname{Rel\_hum}[\Tdb, \Twb, P]}
\noindent Parámetros:
\vardef{\Tdb}temperatura de bulbo seco [\degC]
\vardef{\Twb}temperatura de bulbo humedo [\degC]
\vardef{P}presión ambiental [kPa]
\\Salida:
\vardef{\RH}humedad relativa [fracción o porcentaje]
\ashref.

\begin{align}
    W &= \funcname{Hum\_rat}[\Tdb, \Twb, P] \\
    \RH &= \frac{\funcname{Part\_press}[P, W]}{\funcname{Sat\_press}[\Tdb]}
\end{align}


\subsection{\funcname{Rel\_hum2}[Tdb, W, P]}
\noindent Parámetros:
\vardef{\Tdb}temperatura de bulbo seco [\degC]
\vardef{W}humedad específica [kg/kg dry air]
\vardef{P}presión ambiental [kPa]
\\Salida:
\vardef{\RH}humedad relativa [fracción o porcentaje]
\ashref.

\begin{align}
    \Pw &= \funcname{Part\_press}[P, W]\\
    \Pws &= \funcname{Sat\_press}[\Tdb]\\
    \RH &= \frac{P_w}{\Pws}
\end{align}


\subsection{\funcname{Wet\_bulb}[\Tdb, \RH, P]}
\noindent Parámetros:
\vardef{\Tdb}temperatura de bulbo seco [\degC]
\vardef{\RH}humedad relativa [fracción o porcentaje]
\vardef{P}presión ambiental [kPa]
\\Salida:
\vardef{\Twb}temperatura de bulbo humedo [\degC]
\\Se utiliza un método de iteración Newton-Rhapson para una rápida convergencia

\begin{align}
    W\sub{normal} &= \funcname{Hum\_rat2}[\Tdb, \RH, P] \\
    i &= 0 \\
    T\sub[0]{wb} &= \Tdb \\
    W\sub[0]{new} &= \funcname{Hum\_rat}[\Tdb, {T\sub[0]{wb}}, P]
\end{align}
    
Grado de presición del 0.001\% usando Newton-Rhapson:
\\Mientras que $\left|\dfrac{W\sub[i]{new} - W\sub{normal}}{W\sub{normal}}\right| > 0.00001$:
\begin{align}
	i&=i+1\\
    W\sub[2]{new} &= \funcname{Hum\_rat}[\Tdb, {T\sub[i-1]{wb} - 0.001}, P] \\
    W' &= \frac{W\sub{new}-W\sub[2]{new}  }{0.001} \\
    T\sub[i]{wb} &= T\sub[i-1]{wb} - \frac{W\sub[i-1]{new}-W\sub{normal}}{W'} \\
    W\sub[i]{new} &= \funcname{Hum\_rat}[\Tdb, {T\sub[i]{wb}}, P]
\end{align}
Repite. Al final:
\begin{equation}
    \Twb = T\sub[i]{wb}
\end{equation}


\subsection{\funcname{Enthalpy\_Air\_H2O}[\Tdb, W]}
\noindent Parámetros:
\vardef{\Tdb}temperatura de bulbo seco [\degC]
\vardef{W}humedad específica [kg/kg dry air]
\\Salida:
\vardef{h}entalpía [kJ/kg (aire seco)]
\ashref, SI P6.9 eqn 32

\begin{equation}
    h = 1.006\Tdb + (2501 + 1.86\Tdb)W
\end{equation}


\subsection{\funcname{T\_drybulb\_calc}[h,W]}
\noindent Parámetros:
\vardef{h}entalpía [kJ/kg (aire seco)]
\vardef{W}humedad específica [kg/kg dry air]
\\Salida:
\vardef{\Tdb}temperatura de bulbo seco [\degC]
\\calculo inverso a la entalpía arriba.
\\Nota, el estado 0 para imperial es $\sim0$\degF, 0\% de humedad relativa y 1\,atm\,.
El estado 0 para SI es 0\degC, 0\% de humedad relativa y 1\,atm\,.

\begin{equation}
	\Tdb = \frac{h-2501W}{1.006+1.86W}
\end{equation}
    

\subsection{\funcname{Dew\_point}[P, W]}
\noindent Parámetros:
\vardef{P}presión ambiental [kPa]
\vardef{W}humedad específica [kg/kg dry air]
\\Salida:
\vardef{\Tdp}temperatura de punto de rocío [\degC]
\ashref, page 6.9 equation 39 y 40
\\Válido para punts de rocío inferiores a 93\degC

\begin{align}
    c_{14} &= 6.54 \\
    c_{15} &= 14.526 \\
    c_{16} &= 0.7389 \\
    c_{17} &= 0.09486 \\
    c_{18} &= 0.4569 \\
    \Pw &= \funcname{Part\_press}[P, W] \\
    \alpha &= \ln\Pw \\
    T\sub[1]{dp} &= c_{14} + c_{15}\alpha + c_{16}\alpha^2 + c_{17}\alpha^3 + c_{18}\Pw^{0.1984} \\
    T\sub[2]{dp} &= 6.09 + 12.608\alpha + 0.4959\alpha^2
\end{align}
Si $T\sub[1]{dp}\ge0$:
\begin{equation}
	\Tdp=T\sub[1]{dp}
\end{equation}
Si no:
\begin{equation}
	\Tdp=T\sub[2]{dp}
\end{equation}


\subsection{\funcname{Dry\_Air\_Density}[P, Tdb, W]}
\noindent Parámetros:
\vardef{P}presión ambiental [kPa]
\vardef{\Tdb}temperatura de bulbo seco [\degC]
\vardef{W}humedad específica [kg/kg dry air]
\\Salida:
\vardef{\Rda}densidad de aire seco [kg aire seco/$\text{m}^3$]
\ashref, page 6.8 equation 28

\begin{align}
	R\sub{da} &= 287.055 \\
	T &= \Tdb+273.15 \\
	\Rda &= 1000\frac{P}{R\sub{da}T(1+1.6078W)}
\end{align}

\newcommand\type{\sub{type}}
\newcommand\val{\sub{val}}
\subsection{\funcname{psych}[P,x\type,x\val,y\type,y\val, z\type, u\type=\text{``\texttt{Imp}''}]}

Comprueba que $x\type$ esté entre ``$\Tdb$'', ``$W$'' y ``$h$'' (temperatura de bulbo seco, humedad específica o entalpía).
De lo contrario retornará un no-valor.

Comprueba que $x\type$ sea diferente de $u\type$.
De lo contrario retornará un no-valor.

En caso de que el sistema ($u\type$) sea métrico (``\texttt{SI}'') lee los datos sin conversión.
\tab Convierte la presión de pascales a kilopascales.
\tab De acuerdo con $x\type$ guarda $x\val$ en la variable adecuada $\Tdb$, $W$ o $h$.
\tab De acuerdo con $y\type$ guarda $y\val$ en la variable adecuada $\Tdb$, $\Twb$, $\Tdp$, $\RH$, $W$ o  $h$.

En caso contrario (asume sistema imperial), lee los datos convirtiéndolos.
\tab Convierte la presión de psi (libras por pulgada cuadrada) a kilopascales.
\tab Si $x\type$ o $y\type$ es temperatura, guarda la respectiva variable $\Tdb$, $\Twb$ o $\Tdp$ convirtiéndo de grados farenheit a celcius.
\tab Si $x\type$ o $y\type$ es entalpía, convierte de Btu por libra a kilojulios por kilogramo y corre el punto 0 (de 0\degF\ a 0\degC).
\tab Si $x\type$ o $y\type$ es humedad relativa or específica, guarda la respectiva variable $\RH$ o $W$ sin cambios.

Si $\Tdb$ no existe ($x\type$ y $y\type$ son entropía y humedad específica), calcula $\Tdb=\funcname{T\_drybulb\_calc}[h,W]$.

Si $z\type$ es ``$\RH$'' o ``$\Twb$'', calcula $\RH$:
\tab Si $\Twb$ fue dado: $\RH=\funcname{Rel\_hum}[\Tdb, \Twb, P]$.
\tab Si no, pero $\Tdp$ fue dado: $\RH=\funcname{Sat\_press}[\Tdp]\div\funcname{Sat\_press}[\Tdb]$.
\tab Si no, pero $W$ fue dado: $\RH=\funcname{Part\_press}[P, W]\div\funcname{Sat\_press}[\Tdb]$.
\tab Si no, pero $h$ fue dado: \begin{align}
	W &= -\frac{1.006 \Tdb - h}{2501 + 1.86 \Tdb} \\
	\RH &= \funcname{Part\_press}[P, W]\div\funcname{Sat\_press}[\Tdb]
\end{align}
En caso de otro $z\type$, se busca entonces $W$:
\tab Si $z\type$ no es ``$W$'':
\tab[2]Si $y\type$ es ``$\Twb$'': $W=\funcname{Hum\_rat}[\Tdb, \Twb, P]$
\tab[2]Si $y\type$ es ``$\Tdp$'': $W=0.621945\dfrac{\funcname{Sat\_press}[\Tdp]}{P-\funcname{Sat\_press}[\Tdp]}$
\tab[2]Si $y\type$ es ``$\RH$'': $W=\funcname{Hum\_rat2}[\Tdb, \RH, P]$
\tab[2]Si $y\type$ es ``$h$'': $W=-\dfrac{1.006 \Tdb - h}{2501 + 1.86 \Tdb}$
\tab Si no, indica que la combinación de variables es incorrecta.

En este punto se tienen: $P$, $\Tdb$ y $W$.

Si $z\type$ es ``$\Tdb$'': la respuesta es $\Tdb$.

Si $z\type$ es ``$\Twb$'': la respuesta es $\funcname{Wet\_bulb}[\Tdb, \RH, P]$.

Si $z\type$ es ``$\Tdp$'': la respuesta es $\funcname{Dew\_point}[P, W]$.

Si $z\type$ es ``$\RH$'': la respuesta es $\RH$.

Si $z\type$ es ``$W$'': la respuesta es $W$.

Si $z\type$ es ``$\Pw$'': la respuesta es $1000\cdot\funcname{Part\_press}[P, W]$.

Si $z\type$ es ``\texttt{DSat}'' (grado de saturación): la respuesta es $W\div\funcname{Hum\_rat2}[\Tdb, 1, P]$.

Si $z\type$ es ``$h$'': la respuesta es $\funcname{Enthalpy\_Air\_H2O}[\Tdb, W]$.

Si $z\type$ es ``$s$'' (entropía): la respuesta es inducir un error.

Si $z\type$ es ``\texttt{SV}'' (volumen específico): la respuesta es $1\div\funcname{Dry\_Air\_Density}[P, \Tdb, W]$.

Si $z\type$ es ``\texttt{MAD}'' (densidad): la respuesta es $(1+W)\funcname{Dry\_Air\_Density}[P, \Tdb, W]$.

\noindent Finalmente, si $y\type$ es ``\texttt{Imp}'' (sistema imperial):
\tab Si $z\type$ es temparatura, convierte el valor de salida de \degC\ a \degF.
\tab Si $z\type$ es presión (``$\Pw$''), convierte la salida de pascales a psi.
\tab Si $z\type$ es entalpía, convierte de $\text{kJ}/\text{kg}$ a btu por libra y corre el 0.
\tab si $z\type$ es volúmen específico, convierte de $\text{m}^3/\text{kg}$ a pies cúbicos por libra.
\tab si $z\type$ es densidad, convierte $\text{kg}/\text{m}^3$ a libras por pie cubico.

\subsection{\funcname{main}}
Pide el valor de la presión atmosférica $P$.

Pide el tipo de la primera variable $x\type$.

Pide el valor de la primera variable $x\val$.

Pide el tipo de la segunda variable $y\type$.

Pide el valor de la segunda variable $y\val$.

Pide el tipo de salida $z\type$.

Pide el sistema de medidas ``\texttt{SI}'' o ``\texttt{Imp}'': $u\type$.

Llama a \funcname{psych}[P,x\type,x\val,y\type,y\val, z\type, u\type].

Imprime el resultado.

\end{document}

