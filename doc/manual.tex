\documentclass[letterpaper]{article}
\usepackage{psicret}
\title{\psicret~User Manual}
\author{Chlewey}
\date{2015}
\begin{document}
\maketitle
\begin{abstract}
\psicret\ is a package that include a library and some few interfaces
for making psychrometric calculations.
The interaces include a command line interface, a graphical interface,
and a web (\textsc{php}) interface.

This manual focuses on the command line interface (CLI) and the library.
\end{abstract}
\tableofcontents
\section{Introduction}
\subsection{Psychrometrics}
Psychrometrics or psychrometry or hygrometry are terms used to describe
the field of engineering concerned with the determination of physical
and thermodynamic properties of gas-vapor mixtures. The term derives
from the Greek psuchron (\GR{ψυχρόν}) meaning ``cold'' and metron
(\GR{μέτρον}) meaning ``means of measurement''.\footnote{%
\textit{http://en.wikipedia.org/wiki/Psychrometrics}}
\subsection{Psychrometric chart}
A psychrometric chart is a graph of the thermodynamic parameters of
moist air at a constant pressure, often equated to an elevation relative
to sea level. The \ashrae-style psychrometric chart was pioneered by
Willis Carrier in 1904. It depicts these parameters and is thus a
graphical equation of state.

The ``\mollier'' (Enthalpy - Humidity Mixing Ratio) diagram,
developed by Richard Mollier in 1923, is an alternative psychrometric
chart, preferred by many users in Scandinavia, Eastern Europe, and
Russia.
\section{The library}
\section{The command line}
\subsection{Text output}
\begin{clioptions}
\clioption[m]{si} Uses international metric system (SI or metric) for
  input and output units. This is the default.
\clioption[e]{english} Uses imperial/English units for input and output.
\clioption[p]{pressure}[pres] Sets the pressure in kilopascal or psi.
  a letter suffix might change the default unit:
  \texttt{kpa} kilopascal, \texttt{pa} pascal, \texttt{bar}, \texttt{atm}
  atmosphere, \texttt{tor} torr, \texttt{hg} or \texttt{mmhg} millimeter
  of mercury, \texttt{psi} pounds per square inch.
\clioption[h]{elevation}[height] Sets the elevation in meters or feet
  above sea level. It can accept a unit suffix including \texttt{m} meter,
  \texttt{ft} feet.

If both elevation and pressure parameters are given, it makes calculations
using pressure. (For graphical output both parameters will show in the
chart, but calculations are made using pressure.)

If neither is provided, it assumes sea level (1\,atm pressure).
\clioption[t]{dbt}[temp] Sets the dry bulb temperature. Units are celcius
  or farenheight degrees. Unit suffixes include \texttt{K} for Kelvin,
  \texttt{C} for celcius and \texttt{F} for farenheight. suffixes are not
  case-sensitive.
\clioption[w]{wbt}[temp] Sets the wet bulb temperature.
\clioption[p]{relative}[percentage] Sets the relative humidity.
\clioption[d]{dpt}[temp] Sets the dew point temperature.
\clioption[h]{humidity}[prop] Sets the specific humidity.
\clioption[a]{absolute}[value] Sets the absolute humidity (water vapor density).
\clioption[e]{entalpy}[value] Sets the specific entalpy.
\clioption[v]{volume}[value] Sets the specific volume.
\clioption[r]{ratio}[value] Sets the psychrometric ratio.
\clioption{heat}[value] Sets the humid heat.
\clioption{lang}[language] Sets the output language. Default is English.
  \texttt{\textit{language}} is usually a two-letter code, which refers
  to the file \texttt{bab-\textit{language}.dic} in the program path
  with translations of the different terms and units.
\end{clioptions}

When a text output is required, two parameters should be provided aside
pressure/elevation.

\subsection{Graphical output}
The command line interface can create an \ashrae-style or a \mollier\ 
psychrometric chart for a given air preasure (or height), either empty
or with a marked point.

If one variable (aside pressure/elevetion) is provided, the chart will
highlight that respective line.

If two variables (aside pressure/elevetion) are provided, the chart will
mark the specific point, highlight all relevant lines,
and display all calculated variables in a table.

The following parameters can be used (this settings are for
\ashrae-style charts in landscape mode, \mollier\ charts are in portrait
mode and width and height are traspossed accordingly):

\begin{clioptions}
\clioption[V]{svg}[file] Outputs to \texttt{\textit{file}} (or
  \texttt{\textit{file}.svg} if no extension provided) using scalable
  vector graphics (\textsc{svg}) format.
  If no further commands are issued, this will be a
  $11\,\textrm{inch}\times21\,\textrm{cm}$ output with $2\,\textrm{cm}$
  margins.
\clioption{ps}[file] Outputs to \texttt{\textit{file}} (or
  \texttt{\textit{file}.ps} if no extension provided) using postscript
  format.
  If no further commands are issued, this will be a
  $11\,\textrm{inch}\times21\,\textrm{cm}$ output with $2\,\textrm{cm}$
  margins.
\clioption[R]{png}[file] Outputs to \texttt{\textit{file}} (or
  \texttt{\textit{file}.png} if no extension provided) using portable
  network graphics (\textsc{png}) format.
  If no further commands are issued, this will be a
  $800\times600$ pixels output with $2\,\textrm{cm}$
  margins.
\clioption[W]{width}[x] Set the width of the output file to
  \texttt{\textit{x}} pixels or, if a twoletter unit code is set, to
  that specific length. Default is 800 pixels for raster output
  (\textsc{png}) or 11\,inch (-2\,cm margin at each side) for scalable
  output.
\clioption[H]{height}[y] Set the height of the output file to
  \texttt{\textit{y}} pixels or, if a twoletter unit code is set, to
  that specific length. Default is 600 pixels for raster output
  (\textsc{png}) or 21\,cm (-2\,cm margin at each side) for scalable
  output.
\clioption[D]{dpi}[res] Sets the resolution in dots per inch.
  Default is 90\,dpi\,.
\clioption{dpm}[res] Sets the resolution in dots per meter.
  Default is 3543.307087\,d/m\,.
\clioption[M]{margin}[top,left,bottom,right] Sets the margins for the
  output. The default is zero for raster and 2\,cm at each side for
  scalable formats.
  If no \texttt{\textit{right}} value is provided, it takes it equal to
  \texttt{\textit{left}}. If \texttt{\textit{bottom}} is missing, it
  takes it equal to \texttt{\textit{top}}. If \texttt{\textit{left}}
  is missing, it takes it equal to \texttt{\textit{top}}.
  
  The default units are pixels. If only one of the parameters have a
  twocode unit name, that unit is used for all other margins.
  
  Example: \texttt{-m 1,1.5cm,2} will set left and right margins to
  1.5\,cm, the top margin to 1\,cm and the bottom margin to 2\,cm\,.
\clioption[S]{size}[format] Sets the size for the output using a
  keyword. Common keywords include:
  \texttt{letter} for letter papersize ($11\,\textrm{inch}\times8.5\,\textrm{inch}$), 1\,inch margins.
  \texttt{a4} for A4 papersize ($297\,\textrm{mm}\times210\,\textrm{mm}$), 25\,mm margins.
  \texttt{b4} for B4 papersize ($353\,\textrm{mm}\times250\,\textrm{mm}$), 30\,mm margins.
  \texttt{hd} for 1080p widescreen format ($1920\times1080$ pixels), no margins.
  \texttt{vga} for VGA screen format ($640\times480$ pixels), no margins.
  \texttt{default} for default format, according if it is a scalable or a raster format.
\cliop A shorthand for \texttt{--size=a4}
\cliop B shorthand for \texttt{--size=b4}
\cliop L shorthand for \texttt{--size=letter}
\clioption[T]{title}["the title"] Sets a title for the chart
\clioption[n]{no-lines} Will not highlight the relevant lines if a point is given.
\clioption[N]{no-table} Will not print the table of values if a point is given.
\end{clioptions}
\section{Graphical interface}
\section{Web interface}
\section{Installation}
\section{License}
The following is the text of the license by which this package is distributed.
\verbatiminput{../LICENSE}
\end{document}
